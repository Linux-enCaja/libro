

\chapter{Contenido del Diplomado}

%\section{Temario}

\begin{itemize}

  \large
  \item \textbf{Sistema Operativo Linux}
  \normalsize
    \begin{itemize}
      \item Definiciones y generalidades.
      \item Operaciones b�sicas en Linux.
        \begin{itemize}
          \item Instalaci�n de aplicaciones.
          \item Variables de entorno.
          \item Comandos b�sicos.
        \end{itemize}
    \end{itemize}    


 \large
  \item \textbf{Sistemas Embebidos - Herramientas GNU}
  \normalsize
    \begin{itemize}
      \item Definiciones y generalidades.
      \item Flujo de dise�o de un Sistema Embebido.
        \begin{itemize}
          \item Tareas Hardware
          \item Tareas Software
        \end{itemize}
	  \item Cadena de herramientas GNU para compilaci�n cruzada de sistemas embebidos.
    \end{itemize}

 \large
  \item \textbf{M�quinas de Estado Algor�tmicas (ASM)}
  \normalsize
    \begin{itemize}
      \item Implementaci�n de tareas Hardware usando m�quinas de estado algor�tmicas.
        \begin{itemize}
          \item Multiplicador binario implementado en una ASM.
          \item Divisor binario implementado en una ASM.
        \end{itemize}
      \item Dispositivos l�gicos programables.
      \item Lenguajes de descripci�n.
      \item \textbf{Pr�ctica I: Ejemplos de implementaci�n.}
    \end{itemize}

 \large
  \item \textbf{Sistemas Sobre Silicio (SoC)}
  \normalsize
    \begin{itemize}
      \item Implementaci�n de tareas software usando SoCs.
        \begin{itemize}
          \item Arquitectura del procesador MICO32.
          \item Arquitectura y programaci�n del SoC LM32.
            \begin{itemize}
              \item Herramientas GNU.
              \item Interfaz con los perif�ricos.              
            \end{itemize}
        \end{itemize}
      \item \textbf{Pr�ctica II: Ejemplos de implementaci�n.}
    \end{itemize}

 \large
  \item \textbf{Aplicaciones Utilizando SoC Comerciales}
  \normalsize
    \begin{itemize}
      \item Plataforma de desarrollo AndroidStamp.
        \begin{itemize}
          \item Arquitectura del procesador iMX233.
          \item Generalidades de la tarjeta de desarrollo.
          \item Implementaci�n de programas de ejemplo: Hello World, Blinker.
        \end{itemize}
      \item \textbf{Pr�ctica III: Ejemplos de implementaci�n sobre el iMX233.}
    \end{itemize}

    
    \large
  \item \textbf{Detalles del Sistema Operativo Linux}
  \normalsize
    \begin{itemize}
      \item Definiciones y generalidades del kernel.
      \item Requerimientos HW/SW.
      \item Estructura del c�digo fuente.
      \item Proceso de inicio de Linux.
      \item Drivers de dispositivos.
      \item \textbf{Pr�ctica IV: Configuraci�n, compilaci�n e inicio de un kernel ``a la medida'' utilizando la interfaz de red.}
    \end{itemize}

  \large
  \item \textbf{Sistema de Archivos y M�dulos del Kernel}
  \normalsize
    \begin{itemize}
      \item Estructura de un sistema de archivos.
      \item Archivos importantes: archivos de configuraci�n y archivos de dispositivos.
      \item Sistemas de archivos disponibles: Debian, Buildroot, Openembedded.
      \item Diferencias entre el espacio de usuario y el espacio de kernel.
      \item Definici�n y estructura de un m�dulo kernel.
      \item \textbf{Pr�ctica IV: Configuraci�n, compilaci�n de dos sistemas de archivos: Debian y Buildroot, utilizando NFS.}
      \item \textbf{Pr�ctica V:  Creaci�n de un m�dulo sencillo.}
    \end{itemize}

  \large
  \item \textbf{Dise�o, Fabricaci�n y Montaje de Placas de Circuito Impreso}
  \normalsize
    \begin{itemize}
      \item Dise�o de esquem�ticos y creaci�n de componentes.
      \item Ubicaci�n de componentes pre-ruteo y creaci�n de footprints.
      \item Ruteo manual y autom�tico.
      \item Revisi�n de reglas.
      \item Generac�n de archivos para fabricaci�n y ensamble.
      \item \textbf{Pr�ctica VI:  Dise�o y ruteo de un circuito impreso b�sico.}
    \end{itemize}
\end{itemize}



