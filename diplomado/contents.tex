\chapter{CONTENIDO}

\begin{itemize}

  \large
  \item \textbf{Sistema Operativo Linux}
  \normalsize
    \begin{itemize}
      \item El sistema Operativo Linux
      \item Operaciones b�sicas en Linux
        \begin{itemize}
          \item Instalaci�n de aplicaciones
          \item Variables de entorno
          \item Comandos b�sicos.
        \end{itemize}
    \end{itemize}

    
    


 \large
  \item \textbf{Sistemas Embebidos -- Herramientas GNU}
  \normalsize
    \begin{itemize}
      \item �Q�e es un sistema Embebido?
      \item Flujo de dise�o de un Sistema Embebido.
        \begin{itemize}
          \item Tareas Hardware
          \item Tareas Software
        \end{itemize}
    \end{itemize}

 \large
  \item \textbf{M�quinas de Estado Algor�tmicas (ASM)}
  \normalsize
    \begin{itemize}
      \item Implementaci�n de tareas hardware usando m�quinas de estado algor�tmicas.
        \begin{itemize}
          \item Multiplicador binario implementado en una ASM.
          \item Divisor binario implementado en una ASM.
        \end{itemize}
      \item Dispositivos l�gicos programables.
      \item Lenguajes de descripci�n de hardware verilog y VHDL.
      \item \textbf{Pr�ctica I: Ejemplos de implementaci�n}
    \end{itemize}

 \large
  \item \textbf{Sistemas sobre silicio (SoC)}
  \normalsize
    \begin{itemize}
      \item Implementaci�n de tareas software usando SoCs.
        \begin{itemize}
          \item Arquitectura del Procesador MICO32.
          \item Arquitectura del SoC LM32.
          \item Programaci�n del SoC LM32.
            \begin{itemize}
              \item Interfaz con los perif�ricos.
              \item Herramientas GNU
            \end{itemize}
        \end{itemize}
      \item \textbf{Pr�ctica II: Ejemplos de implementaci�n}
    \end{itemize}

 \large
  \item \textbf{Aplicaciones utilizando SoC comerciales}
  \normalsize
    \begin{itemize}
      \item Plataforma de Desarrollo stamp.
        \begin{itemize}
          \item Arquitectura del procesador iMX233.
          \item Tarjeta de Desarrollo STAMP
          \item Implementaci�n del Programa: Hello World
          \item Implementaci�n del Programa: Blinker
        \end{itemize}
      \item \textbf{Pr�ctica III: Ejemplos de implementaci�n sobre el iMX233 utilizando herramientas GNU}
    \end{itemize}

    
    \large
  \item \textbf{Detalles del Sistema Operativo Linux}
  \normalsize
    \begin{itemize}
      \item Kernel overview
      \item Requerimientos HW/SW
      \item Estructura del c�digo fuente.
      \item Proceso de inicio de linux
      \item Drivers de Dispositivos
      \item M�dulos del Kernel

      \item \textbf{Pr�ctica IV: Configuraci�n, compilaci�n e inicio de un kernel ``a la medida'' utilizando la interfaz de red para descargarlo a la memoria SDRAM}
    \end{itemize}

  \large
  \item \textbf{Sistema de Archivos y M�dulos del Kernel}
  \normalsize
    \begin{itemize}
      \item Estructura de un sistema de archivos.
      \item Archivos Importantes: Archivos de configuraci�n, archivos de dispositivos.
      \item Sistemas de archivos disponibles: Debian, Buildroot, Openembedded.
      \item �Qu� es un m�dulo?
      \item Diferencias entre el espacio de usario y el espacio de kernel.
      \item Estructura de un m�dulo.
      \item \textbf{Pr�ctica IV: Configuraci�n, compilaci�n de 2 sistemas de archivos: Debian y Buildroot, utilizando NFS}
      \item \textbf{Pr�ctica V:  Creaci�n de un M�dulo Sencillo}
    \end{itemize}

% 
%   \large
%   \item \textbf{Interfaces de red}
%   \normalsize
%     \begin{itemize}
%       \item Introducci�n a la programaci�n en Sockets.
%       \item Uso de librer�a libcurl para desarrollo de aplicaciones basadas en web.
%       \item Introducci�n a HTML5 y WEBSOCKETS.
%       \item Protocolos de capa de aplicaci�n (SSH, NTP, PPP)
%     \end{itemize}
% 
%   \large
%   \item \textbf{Desarrollo de aplicaciones Multimedia}
%   \normalsize
%     \begin{itemize}
%       \item Introducci�n a ALSA.
%       \item Introducci�n a QT.
%     \end{itemize}

  \large
  \item \textbf{Dise�o, fabricaci�n y montaje de placas de circuito impreso}
  \normalsize
    \begin{itemize}
      \item XXXXX
      \item XXXXX
      \item XXXXX
      \item XXXXX
      \item XXXXX
    \end{itemize}

\end{itemize}
